\documentclass[11pt]{scrartcl} % Font size
\input{../structure.tex} % Include the file specifying the document structure and custom commands

%----------------------------------------------------------------------------------------
%	TITLE SECTION
%----------------------------------------------------------------------------------------

\title{
	\normalfont\normalsize
	\textsc{Harvard Privacy Tools Project}\\ % Your university, school and/or department name(s)
	\vspace{25pt} % Whitespace
	\rule{\linewidth}{0.5pt}\\ % Thin top horizontal rule
	\vspace{20pt} % Whitespace
	{\huge Median Sensitivity Proofs}\\ % The assignment title
	\vspace{12pt} % Whitespace
	\rule{\linewidth}{2pt}\\ % Thick bottom horizontal rule
	\vspace{12pt} % Whitespace
}

% \author{\LARGE} % Your name

\date{\normalsize\today} % Today's date (\today) or a custom date

\begin{document}

\maketitle

\begin{definition}
The sample median of a database $X = (x_1, \hdots, x_n)$ is given by
\[ f(X) = \frac{\tilde{x}_l + \tilde{x}_u}{2} \]
where $l = \lfloor \frac{n+1}{2} \rfloor$ and $u = \lceil \frac{n+1}{2} \rceil$ and
$\tilde{X} = (\tilde{x}_1, \hdots, \tilde{x}_n)$ is the sorted version of $X$.
\end{definition}

\section{Neighboring Definition: Change One}

% l1 sensitivity
\subsection{$\ell_1$-sensitivity}
\begin{theorem}
	Let the database $X$ have elements bounded above by $M$ and below by $m$.
	Then the global sensitivity of the median is
	\[
		\Delta f(\cdot) =
			\begin{cases}
				\frac{M - m}{2}, \text{ if } n \equiv 0 \pmod{2} \\
				M-m \text{ if } n \equiv 1 \pmod{2}
			\end{cases}
	\]
\end{theorem}

\begin{proof}
	First consider the case where $n$ is such that $n \equiv 0 \pmod{2}$. Then our worst-case scenario is that exactly
	half of the data elements in $X$ are $m$ and the other half are $M$. In this case,
	$f(X) = \frac{m + M}{2}$. \newline

	Now consider an $X'$ which is identical to $X$ but with one element switched from
	$m$ to $M$ WLOG.\footnote{The result holds if it is switched from $M$ to $m$.}
	Then we have that $f(X') = M$. Because we are looking at our worst-case pairing of $X,X'$,
	we know that
	\[ \Delta f(\cdot) = \max_{X,X'} |f(X') - f(X)| = \big\vert M - \frac{m+M}{2} \big\vert = \frac{M-m}{2}. \]

	Now consider the case where $n$ is such that $n \equiv 1 \pmod{2}$.
	Then our worst-case scenario is that $\frac{n-1}{2}$ of the elements of $X$ are $m$,
	$\frac{n-1}{2}$ of the elements of $X$ are $M$, and the remaining element of $X$ is $m$ (WLOG).
	In this setting, $f(X) = m$.
	Let $X'$ be identical to $x$ but with one element switched from $m$ to $M$.
	Then $f(X') = M$ and we have that
	\[ \Delta f(\cdot) = \max_{X,X'} |f(X') - f(X)| = |M - m|. \]
\end{proof}

% l2 sensitivity
\subsection{$\ell_2$-sensitivity}
\begin{theorem}
	Let the database $X$ have elements bounded above by $M$ and below by $m$.
	Then the global sensitivity of the median is
	\[
		\Delta f(\cdot) =
			\begin{cases}
				\left( \frac{M - m}{2} \right)^2, \text{ if } n \equiv 0 \pmod{2} \\
				(M-m)^2 \text{ if } n \equiv 1 \pmod{2}
			\end{cases}
	\]
\end{theorem}

\begin{proof}
	The logic follows exactly from the proof of the $\ell_1$ sensitivity,
	just with the norm in the last line of each statement switched from 1 to 2.
\end{proof}

\section{Neighboring Definition: Add/Drop One}

% l1 sensitivity
\subsection{$\ell_1$-sensitivity}

% l2 sensitivity
\subsection{$\ell_2$-sensitivity}

\end{document}