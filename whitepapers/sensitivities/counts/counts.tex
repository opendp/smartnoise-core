\documentclass[11pt]{scrartcl} % Font size
\input{../structure.tex} % Include the file specifying the document structure and custom commands

%----------------------------------------------------------------------------------------
%	TITLE SECTION
%----------------------------------------------------------------------------------------

\title{
	\normalfont\normalsize
	\textsc{Harvard Privacy Tools Project}\\ % Your university, school and/or department name(s)
	\vspace{25pt} % Whitespace
	\rule{\linewidth}{0.5pt}\\ % Thin top horizontal rule
	\vspace{20pt} % Whitespace
	{\huge Count Sensitivity Proofs}\\ % The assignment title
	\vspace{12pt} % Whitespace
	\rule{\linewidth}{2pt}\\ % Thick bottom horizontal rule
	\vspace{12pt} % Whitespace
}

% \author{\LARGE} % Your name

\date{\normalsize\today} % Today's date (\today) or a custom date

\begin{document}

\maketitle

\begin{definition}
Let $\mathcal{X}$ be the universe of possible rows (individuals) and let $I: \mathcal{X} \rightarrow \{0,1\}$ be a predicate on rows. Let $x \in \mathcal{X}^n$ be a dataset. Then a count over $x$ is defined as 
$$ q(x) = \sum_{i=1}^n I(x_i).$$
\end{definition}

\begin{definition}
Let $q_1, \ldots, q_k$ be a series of counts with predicates $I_1, \ldots, I_k$. These counts are disjoint for every row in the database if only one of them evaluates to 1. In other words, they are disjoint if $\forall x_i \in \mathcal{X},$ 
$$ \sum_{j=1}^k I_j(x_i) = 1.$$
\end{definition}

\section{Neighboring Definition: Change One}

% l1 sensitivity
\subsection{$\ell_1$-sensitivity}

\begin{theorem}
\label{thm:change1L1}
A single count query has sensitivity 1. A series of $k$ disjoint counts has sensitivity 2.
\end{theorem}

\begin{proof}
Let $q$ be a count query with predicate $I$, and let $x'$ be equal to $x$ with point $x_i$ changed to $x_i'$. Then,

\begin{align*}
\left\vert q(x') - q(x) \right\vert &= \left\vert \sum_{j=1}^n I(x_j') - \sum_{j=1}^n I(x_j) \right\vert \\
	&= \left\vert \sum_{\{ i \in [n] \vert i \ne j\}} I(x_j) + I(x_i') - \sum_{\{ i \in [n] \vert i \ne j\}} I(x_j) - I(x_i) \right\vert \\
	&= \left\vert I(x_i') - I(x_i) \right\vert \\
	& \le 1.
\end{align*}

Consider a series of $k$ disjoint count queries $\mathbf{q} = \{q_1, \ldots, q_k\}$ on the same databases $x$ and $x'$. Note that since the counts are disjoint, $x_i$ and $x_i'$ can at most each increment a single one of the $k$ counts by 1. Call the count that $x_i$ impacts $q_i$, and the count that $x_i'$ impacts $q_j$. Then,

\begin{align*}
\left\vert \mathbf{q}(x) - \mathbf{q}(x') \right\vert &= \left\vert \left(q_1(x) - q_1(x')\right) + \ldots + \left(q_k(x) - q_k(x')\right) \right\vert \\
	&= \left\vert \right(q_i(x) - q_i(x')\left) + \right(q_i(x) - q_j(x')\left) \right\vert \\
	&\le 2
\end{align*}
\end{proof}

% l2 sensitivity
\subsection{$\ell_2$-sensitivity}

\begin{theorem}
A single count query has sensitivity 1. A series of $k$ disjoint counts has sensitivity 2.
\end{theorem}

\begin{proof}
From the proof of Theorem \ref{thm:change1L1}, the difference between counts on two neighboring databases is at most 1. Squaring this gives the same value. For a series of $k$ disjoint counts,
\begin{align*}
\left\vert \mathbf{q}(x) - \mathbf{q}(x') \right\vert_2 &= \left\vert \left(q_1(x) - q_1(x')\right)^2 + \ldots + \left(q_k(x) - q_k(x')\right)^2 \right\vert \\
 & \le \left\vert 1^2 + 1^2 \right\vert \\
 &= 2.
 \end{align*}
\end{proof}

\section{Neighboring Definition: Add/Drop One}
% l1 sensitivity
\subsection{$\ell_1$-sensitivity}
\begin{theorem}
\label{thm:addDropL1}
A single count query has sensitivity 1. A series of $k$ disjoint counts also has sensitivity 1.
\end{theorem}

\begin{proof}
Let $q$ be a count query with predicate $I$, and let $x'$ be equal to database $x$ with point $x_i$ removed. Then 
\begin{align*}
\left \vert q(x) - q(x') \right\vert &= \left\vert \sum_{j=1}^n I(x_j) - \sum_{\{ i \in [n] \vert i \ne j\}} I(x_j) \right\vert\\
	&= \left\vert \sum_{\{ i \in [n] \vert i \ne j\}} I(x_j) + I(x_i) - \sum_{\{ i \in [n] \vert i \ne j\}} I(x_j) \right\vert\\
	&\le 1.
\end{align*}

A nearly identical argument holds for adding a point. \\

Consider a series of $k$ disjoint count queries $\mathbf{q} = \{q_1, \ldots, q_k\}$ and consider database $x'$ equal to database $x$ with point $x_i$ removed. Note that only a single one of the $k$ queries will be affected by the change from $x$ to $x'$, so
\begin{align*}
\left\vert \mathbf{q}(x) - \mathbf{q}(x') \right\vert &= \left\vert \left(q_1(x) - q_1(x')\right) + \ldots + \left(q_k(x) - q_k(x')\right) \right\vert \\
	&\le 1.
\end{align*}
The same argument holds for $x'$ equal to $x$ with a single point added.
\end{proof}
% l2 sensitivity
\subsection{$\ell_2$-sensitivity}
\begin{theorem}
A single count query has sensitivity 1. A series of $k$ disjoint counts also has sensitivity 1.
\end{theorem}

\begin{proof}
Squaring the sensitivity bounds from Theorem \ref{thm:addDropL1} gives 1 as an upper bound on the $\ell_2$ sensitivity for both a single count and a series of $k$ disjoint counts.
\end{proof}
% \bibliographystyle{alpha}
% \bibliography{mean}

\end{document}