\documentclass[11pt]{scrartcl} % Font size
\input{../structure.tex} % Include the file specifying the document structure and custom commands

%----------------------------------------------------------------------------------------
%	TITLE SECTION
%----------------------------------------------------------------------------------------

\title{
	\normalfont\normalsize
	\textsc{Harvard Privacy Tools Project}\\ % Your university, school and/or department name(s)
	\vspace{25pt} % Whitespace
	\rule{\linewidth}{0.5pt}\\ % Thin top horizontal rule
	\vspace{20pt} % Whitespace
	{\huge Count Sensitivity Proofs}\\ % The assignment title
	\vspace{12pt} % Whitespace
	\rule{\linewidth}{2pt}\\ % Thick bottom horizontal rule
	\vspace{12pt} % Whitespace
}

% \author{\LARGE} % Your name

\date{\normalsize\today} % Today's date (\today) or a custom date

\begin{document}

\maketitle

\begin{definition}
Let $\mathcal{X}$ be the universe of possible rows (individuals) and let $I: \mathcal{X} \rightarrow \{0,1\}$ be a predicate on rows. Let $x \in \mathcal{X}^n$ be a dataset. Then a count over $x$ is defined as 
$$ q(x) = \sum_{i=1}^n I(x_i).$$
\end{definition}

\begin{definition}
Let $q_1, \ldots, q_k$ be a series of counts with predicates $I_1, \ldots, I_k$. These counts are disjoint for every row in the database, only one of them evaluates to 1. In other words, they are disjoint if $\forall x_i \in \mathcal{X},$ 
$$ \sum_{j=1}^k I_j(x_i) = 1.$$
\end{definition}

\section{Neighboring Definition: Change One}

% l1 sensitivity
\subsection{$\ell_1$-sensitivity}

\begin{theorem}
\label{thm:change1L1}
A single count query has sensitivity 1. A series of $k$ disjoint counts has sensitivity 2.
\end{theorem}

\begin{proof}
Let $q$ be a count query with predicate $I$, and let $x'$ be equal to $x$ with point $x_i$ changed to $x_i'$. Then $I(x_i)$ will change by at most 1, and since no other $x_j$ with $i \ne j$ is changed, this term in $q$ is the only thing affected. Thus, the sensitivity of a single query is bounded by 1.

Consider $k$ disjoint count on the same databases $x$ and $x'$. Since they are disjoint, only one of the $k$ counts is influenced by $x_i$, and only one of the counts is affected by $x_i'$. Since each query can be affected by at most 1 by a single data point by the above logic, in total there will be a change of at most 2.
\end{proof}

% l2 sensitivity
\subsection{$\ell_2$-sensitivity}

\begin{theorem}
A single count query has sensitivity 1. A series of $k$ disjoint counts has sensitivity 4.
\end{theorem}

\begin{proof}
From the proof of Theorem \ref{thm:change1L1}, the difference between counts on two neighboring databases is at most 1. Squaring this gives the same value. Similarly, the difference between the results of $k$ disjoint counts on two neighboring databases is 2, and squaring this gives 4. 
\end{proof}

\section{Neighboring Definition: Add/Drop One}
% l1 sensitivity
\subsection{$\ell_1$-sensitivity}

% l2 sensitivity
\subsection{$\ell_2$-sensitivity}

% \bibliographystyle{alpha}
% \bibliography{mean}

\end{document}