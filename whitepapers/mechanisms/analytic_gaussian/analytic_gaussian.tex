\documentclass[11pt]{scrartcl} % Font size
\input{../structure.tex} % Include the file specifying the document structure and custom commands

%----------------------------------------------------------------------------------------
%	TITLE SECTION
%----------------------------------------------------------------------------------------

\title{
	\normalfont\normalsize
	\textsc{Harvard Privacy Tools Project}\\ % Your university, school and/or department name(s)
	\vspace{25pt} % Whitespace
	\rule{\linewidth}{0.5pt}\\ % Thin top horizontal rule
	\vspace{20pt} % Whitespace
	{\huge Notes: Analytic Gaussian Mechanism}\\ % The assignment title
	\vspace{12pt} % Whitespace
	\rule{\linewidth}{2pt}\\ % Thick bottom horizontal rule
	\vspace{12pt} % Whitespace
}

\author{} % Your name

\date{} % Today's date (\today) or a custom date

\begin{document}

\maketitle

% CC Note: Not actually sure that I understand the tolerance

% \section{Overview}
% We are currently working toward adding the Analytic Gaussian Mechanism \cite{BW18} to the library. 
% Our \href{https://github.com/opendifferentialprivacy/whitenoise-core/blob/cc_add_analytic_gaussian/runtime-rust/src/utilities/analytic_gaussian.rs}{implementation}
% is meant to be a very close replica of Borja's \href{https://github.com/BorjaBalle/analytic-gaussian-mechanism}{python implementation}, which we understand to be a nearly 
% complete roadmap for implementing the mechanism. We discuss the single missing piece below.

% \section{Problem Intro}
% The Analytic Gaussian Mechanism requires calculation of one of two values, $v^*$ or $u^*$, which is later used 
% to calculate the variance of the noise distribution $\mathcal{N}(0, \sigma^2 I)$ for the mechanism. Borja's algorithm uses binary search
% to find the value up to a given error tolerance, which is set by the user. Our goal is to expose to the user how this 
% error tolerance translates to error tolerance on the quantity they actually care about, $\sigma$.


\bibliographystyle{alpha}
\bibliography{analytic_gaussian}

\end{document}