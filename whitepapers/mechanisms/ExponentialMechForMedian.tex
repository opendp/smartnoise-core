\documentclass[11pt]{scrartcl} % Font size
\input{structure.tex} % Include the file specifying the document structure and custom commands

%----------------------------------------------------------------------------------------
%	TITLE SECTION
%----------------------------------------------------------------------------------------

\title{
	\normalfont\normalsize
	\textsc{Harvard Privacy Tools Project}\\ % Your university, school and/or department name(s)
	\vspace{25pt} % Whitespace
	\rule{\linewidth}{0.5pt}\\ % Thin top horizontal rule
	\vspace{20pt} % Whitespace
	{\huge The Exponential Mechanism for Medians}\\ % The assignment title
	\vspace{12pt} % Whitespace
	\rule{\linewidth}{2pt}\\ % Thick bottom horizontal rule
	\vspace{12pt} % Whitespace
}

% \author{\LARGE} % Your name

\date{\normalsize\today} % Today's date (\today) or a custom date

\begin{document}
\maketitle

\section{The Exponential Mechanism}

Sometimes, the global sensitivity of a function is too great, so the Laplace mechanism will not produce meaningful results. The median is one such function. In many cases, the \textit{Exponential mechanism} is an alternate approach that gives reasonable utility.\footnote{This is not the \textit{only} advantage of the exponential mechanism. It is a way to compute differentially private queries on non-numeric data, unlike the Laplace mechanism it does not assume that the probability of outputting a response ought to be symmetric about the true response, etc.} Introduced in 2007 by McSherry and Talwar, the exponential mechanism posits that for a given database, users prefer some outputs over others. That those preferences may be encapsulated with a utility score, where a high utility score indicates a higher preference for that output. The exponential mechanism releases outputs with probability proportional (in the exponent) to the utility score and the sensitivity of the utility function. 

\begin{definition}  
Let $\mathcal{X}$ be a space of databases and let $[m,M]$ be an arbitrary range. Let $u: \mathcal{X} \times [m,M] \rightarrow \mathbb{R}$ be a utility function, which maps pairs of databases and outputs to a utility score. Let $\Delta u$ be the sensitivity of $u$ with respect to the database argument. The exponential mechanism outputs $r \in [m,M]$ with probability proportional to $\exp\left(\frac{\varepsilon u(x,r)}{2 \Delta u}\right)$ \cite{mcsherry2007mechanism, dwork2014algorithmic}.\footnote{The original definition is from \cite{mcsherry2007mechanism}, but here we state the version rewritten in \cite{dwork2014algorithmic} as it is slightly clearer.}
\end{definition}

\begin{theorem}
The exponential mechanism preserves $(\varepsilon,0)$-differential privacy \cite{mcsherry2007mechanism, dwork2014algorithmic}.\footnote{As written in \cite{mcsherry2007mechanism}, the mechanism actually preserves $(2\varepsilon\Delta u,0)$-differential privacy; the main difference in the $\cite{dwork2014algorithmic}$ version is that it has the extra factor of $2\Delta u$ to avoid these extra terms.}
\end{theorem}

Note that the exponential mechanism may not be tractable in many cases, as it assumes the existence of a utility function, and even if one exists it may not be efficiently computable. 

\section{An Exponential Mechanism for Quantiles}

\subsection{Defining a sensible utility function}
%
First, consider the case where the desired quantile is the median. Note that a user will prefer an output that is closer to the true median over one that is further away. Let $x$ be an (ordered) data set, and let $r$ be a possible median output by our mechanism. Note that if $r$ is exactly the median, there should be the same number of points in $x$ to the left and to the right of $r$. As $r$ decreases or increases, then the distance between the number of points to the right of $r$ and to the left of $r$ will increase. So, the distance between the number of points to the left and the number of points to the right of $r$ encapsulates how close the output is to the true median. \\

Slightly more formally, let $\#(x<r)$ denote the number of points to the left of $r$ in database $x$ and let $\#(x<r)$ denote the number of points to the right of $r$ in database $x$. Define the utility function $u$ as

\begin{equation}
u(x,r) \ \left\vert \#(x<R) - \#(x > R)\right\vert.
\end{equation}

To generalize this to an arbitrary quantile, let $N$ be the size of $x$ and let $\alpha \in (0,1)$ indicate the desired quantile. Then, we can modify our initial utility function as follows:

\begin{equation}
\label{eq:quant-util}
u(x,r) = \max(\alpha, (1-\alpha))N - \left\vert (1-\alpha) \#(x<r) - \alpha \# (x>r) \right\vert .
\end{equation}

Say for example that $\alpha = 0.25$. Then, if $r$ is exactly at the first quantile, 
$\#(x<r) $ will be the count of one fourth of the data and $\# (x>r)$ will be three fourths of the data; 

\subsection{Sensitivity of the utility function}

\subsubsection{Neighboring Definition: Change One}

\begin{theorem}
Let $u$ be defined as in Eq.~\ref{eq:quant-util}. The $\ell_1$-sensitivity of $u$ in the change-one model is bounded above by 1. 
\end{theorem}

\begin{proof}
Say $X$ and $X'$ are neighboring databases which differ at some data point. Let $\Delta u$ denote the $\ell_1$-sensitivity of $u(\cdot, \cdot)$ with respect to the space of databases. 

Let $C_1 = \#(Z<x), C_2 = \#(Z>x)$. Worst case, $C_1$ increases by 1 and $C_2$ decreases by 1.
Then 

\begin{align*}
\Delta u &= \left\vert \left\vert (1-\alpha) (C_1 + 1) - \alpha (C_2-1) \right\vert - \left\vert (1-\alpha) C_1 - \alpha C_2 \right\vert \right\vert \\
 &\le \left\vert (1-\alpha) (C_1 + 1) - \alpha (C_2-1) - (1-\alpha) C_1 + \alpha C_2 \right\vert  \\
 &\le \left\vert C_1 + 1 - \alpha C_1 - \alpha - \alpha C_2 + \alpha - C_1 + \alpha C_2 + \alpha C_2 \right\vert  \\
 &= 1
\end{align*}

If instead $C_2$ increases by 1 and $C_1$ decreases by 1, the result is identical (negative sign falls out).

% note that this is definitely worse than only one of them happening cause then you're just in the add/drop one setting and you get max(alpha, 1-alpha).
\end{proof}

\subsubsection{Neighboring Definition: Add/Drop One}

\begin{theorem}
Let $u$ be defined as in Eq.~\ref{eq:quant-util}. The $\ell_1$-sensitivity of $u$ in the add/drop one model is bounded above by $\max(\alpha, 1-\alpha)$.. 
\end{theorem}

\begin{proof}
If we add 1, there are two worse-cases: $C_2$ increases by 1, nothing happens to $C_2$.
\begin{align*}
\Delta u &= \left\vert (1-\alpha) (C_1 + 1) - \alpha (C_2) \right\vert - \left\vert (1-\alpha) C_1 - \alpha C_2 \right\vert \\
 &\le \left\vert (1-\alpha) (C_1 + 1) - \alpha (C_2) - (1-\alpha) C_1 + \alpha C_2 \right\vert \\
&= \left\vert C_1 + 1 - \alpha C_1 - \alpha - \alpha C_2 - C_1 + \alpha C_1 + \alpha C_2 \right\vert \\
&= 1 - \alpha
\end{align*}
b. nothing happens to $C_1, C_2$ increases by 1
\begin{align*}
\Delta u &= \left\vert (1-\alpha) (C_1) - \alpha (C_2 + 1) \right\vert - \left\vert (1-\alpha) C_1 - \alpha C_2 \right\vert \\
&\le \left\vert C_1 -\alpha C_1 - \alpha C_2 - \alpha - C_1 + \alpha C_1 + \alpha C_2 \right\vert \\
&= \alpha 
\end{align*}

Subtracting a point gives you same thing but with some negative signs inside the absolute values that come out in the wash.
\end{proof}

\subsection{The Normalization Factor}

\bibliographystyle{C_1}
\nocite{*}
\bibliography{expMechMedian}
\end{document}