\documentclass[11pt]{scrartcl} % Font size
\input{structure.tex} % Include the file specifying the document structure and custom commands

%----------------------------------------------------------------------------------------
%	TITLE SECTION
%----------------------------------------------------------------------------------------

\title{
	\normalfont\normalsize
	\textsc{Harvard Privacy Tools Project}\\ % Your university, school and/or department name(s)
	\vspace{25pt} % Whitespace
	\rule{\linewidth}{0.5pt}\\ % Thin top horizontal rule
	\vspace{20pt} % Whitespace
	{\huge Laplace Mechanism Accuracy}\\ % The assignment title
	\vspace{12pt} % Whitespace
	\rule{\linewidth}{2pt}\\ % Thick bottom horizontal rule
	\vspace{12pt} % Whitespace
}

\date{\normalsize\today} % Today's date (\today) or a custom date

\begin{document}

\maketitle


\begin{definition}
Let $z$ be the true value of the statistic and let $X$ be the random variable the noisy release is drawn from. Let $\alpha$ be the statistical significance level, and let $Y = \vert X-z \vert.$ Then, accuracy $a$ for a given $\alpha \in [0,1]$ is the $a$ s.t.
$$ \alpha = \pr[Y > a].$$
\end{definition}

\begin{theorem}
The accuracy of an $\epsilon$-differentially private release from the Laplace mechanism on a function with $\ell_1$-sensitivity $\Delta_1$, at statistical significance level $\alpha$ is
$$ a = \frac{\Delta_1}{\epsilon}\ln(1/\alpha).$$
\end{theorem}

\begin{proof}
Recall the definition of the Laplace mechanism, which adds Laplace noise proportional to $\Delta_1/\epsilon$ to the true query responses \cite{DMNS06}. The probability density function $f$ of the Laplace distribution, for $x \sim X$ with location parameter $\mu$ and scaling parameter $\lambda$ is defined to be

$$ f(x) = \frac{1}{2\lambda}e^{\frac{-\vert x-\mu \vert}{ \lambda}}$$

Then, since the pdf $g$ of $Y$ is the same as the folded pdf of $X$, shifted over by $\mu$ and doubled,

$$ g(y) = \frac{1}{\lambda}e^{-y/\lambda}.$$\

Then, 

\begin{align*}
\alpha &= \pr[Y>a] \\
	&= 1 - \pr[Y\le a]\\
	&= 1 - \int_{-\infty}^a g(y) dy \\
	&= 1 - \int_0^a g(y) dy \\
	&= 1 - (1-e^{-a/\lambda})
\end{align*}

Solving for $a$ gives $a = \lambda \ln (1/\alpha)$. Then, since 

$$\lambda = \frac{\Delta_1}{\epsilon},$$

$$ a = \frac{\Delta_1}{\epsilon}\ln(1/\alpha).$$
\end{proof}

\bibliographystyle{alpha}
\bibliography{accuracies}

\end{document}