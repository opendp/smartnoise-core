\documentclass[11pt]{scrartcl} % Font size
\input{structure.tex} % Include the file specifying the document structure and custom commands

%----------------------------------------------------------------------------------------
%	TITLE SECTION
%----------------------------------------------------------------------------------------

\title{
	\normalfont\normalsize
	\textsc{Harvard Privacy Tools Project}\\ % Your university, school and/or department name(s)
	\vspace{25pt} % Whitespace
	\rule{\linewidth}{0.5pt}\\ % Thin top horizontal rule
	\vspace{20pt} % Whitespace
	{\huge Gaussian Mechanism Accuracy}\\ % The assignment title
	\vspace{12pt} % Whitespace
	\rule{\linewidth}{2pt}\\ % Thick bottom horizontal rule
	\vspace{12pt} % Whitespace
}

\date{\normalsize\today} % Today's date (\today) or a custom date

\begin{document}

\maketitle

\begin{definition}
Let $z$ be the true value of the statistic and let $X$ be the random variable the noisy release is drawn from. Let $\alpha$ be the statistical significance level, and let $Y = \vert X-z \vert.$ Then, accuracy $a$ for statistical significance level $\alpha$ is the $a$ s.t.
$$ \alpha = \pr[Y > a].$$
\end{definition}

\begin{theorem}
The accuracy of an $(\epsilon,\delta)$-differentially private release from the Gaussian mechanism on a function with $\ell_2$-sensitivity $\Delta_2$ at statistical significance level $\alpha$ is
$$ a = \frac{2 \Delta_2}{\epsilon} \sqrt{\ln \left( \frac{1.25}{\delta} \right)} \hspace{0.1cm} \text{erf}^{-1} (1 - \alpha).$$
\end{theorem}

\begin{proof}
Note that
\begin{align*}
\alpha &= 1 - P[Y \le a] \\
	&= 1 - 2 \int_0^a f(y) dy \\
	&= 1 - \frac{2}{\sigma \sqrt{2\pi}} \int_0^a e^{-\frac{1}{2}\left( \frac{y}{\sigma}\right)^2} dy \\
	&= 1 - \text{erf}\left( \frac{a}{\sigma \sqrt{2}}\right).
\end{align*},

where the last two lines are taken from the definitions of the Gaussian distribution.

Recall from the definition of the Gaussian mechanism that it adds Gaussian noise to queries with standard deviation
$\sigma \ge c \Delta_2 / \epsilon$, where $c^2 \ge 2 \ln \left( \frac{1.25}{\delta} \right)$ \cite{DR14}.\footnote{In the formulation of the Gaussian mechanism in Dwork \& Roth, they say that $c^2 \ge 2 \ln \left( \frac{1.25}{\delta} \right)$. However, this bound comes from a tighter bound in their proof on p.264, which requires that $c^2 > 2 \ln (\sqrt{2 e^{8/9}/\pi}(1/\delta))$. Since $\sqrt{2 e^{8/9}/\pi} < 1.25$, the bound can safely be made tight.} Setting $\sigma$ into its minimum value, plugging it into the above expression and solving for $a$ then gives

\begin{align*}
a &= \sigma \sqrt{2} \hspace{0.1cm} \text{erf}^{-1} (1 - \alpha)\\
	&= \frac{\Delta_2 \sqrt{2}}{\epsilon} \sqrt{ 2 \ln \left( \frac{1.25}{\delta} \right)} \hspace{0.1cm} \text{erf}^{-1} (1 - \alpha)\\
	&= \frac{2 \Delta_2}{\epsilon} \sqrt{\ln \left( \frac{1.25}{\delta} \right)} \hspace{0.1cm} \text{erf}^{-1} (1 - \alpha)
\end{align*}
\end{proof}

\bibliographystyle{alpha}
\bibliography{accuracies}

\end{document}